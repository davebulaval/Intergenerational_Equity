\documentclass[]{article}
\usepackage{lmodern}
\usepackage{amssymb,amsmath}
\usepackage{ifxetex,ifluatex}
\usepackage{fixltx2e} % provides \textsubscript
\ifnum 0\ifxetex 1\fi\ifluatex 1\fi=0 % if pdftex
  \usepackage[T1]{fontenc}
  \usepackage[utf8]{inputenc}
\else % if luatex or xelatex
  \ifxetex
    \usepackage{mathspec}
  \else
    \usepackage{fontspec}
  \fi
  \defaultfontfeatures{Ligatures=TeX,Scale=MatchLowercase}
\fi
% use upquote if available, for straight quotes in verbatim environments
\IfFileExists{upquote.sty}{\usepackage{upquote}}{}
% use microtype if available
\IfFileExists{microtype.sty}{%
\usepackage{microtype}
\UseMicrotypeSet[protrusion]{basicmath} % disable protrusion for tt fonts
}{}
\usepackage[margin=1in]{geometry}
\usepackage{hyperref}
\hypersetup{unicode=true,
            pdftitle={Recommandations de lectures},
            pdfauthor={David Beauchemin},
            pdfborder={0 0 0},
            breaklinks=true}
\urlstyle{same}  % don't use monospace font for urls
\usepackage{graphicx,grffile}
\makeatletter
\def\maxwidth{\ifdim\Gin@nat@width>\linewidth\linewidth\else\Gin@nat@width\fi}
\def\maxheight{\ifdim\Gin@nat@height>\textheight\textheight\else\Gin@nat@height\fi}
\makeatother
% Scale images if necessary, so that they will not overflow the page
% margins by default, and it is still possible to overwrite the defaults
% using explicit options in \includegraphics[width, height, ...]{}
\setkeys{Gin}{width=\maxwidth,height=\maxheight,keepaspectratio}
\IfFileExists{parskip.sty}{%
\usepackage{parskip}
}{% else
\setlength{\parindent}{0pt}
\setlength{\parskip}{6pt plus 2pt minus 1pt}
}
\setlength{\emergencystretch}{3em}  % prevent overfull lines
\providecommand{\tightlist}{%
  \setlength{\itemsep}{0pt}\setlength{\parskip}{0pt}}
\setcounter{secnumdepth}{0}
% Redefines (sub)paragraphs to behave more like sections
\ifx\paragraph\undefined\else
\let\oldparagraph\paragraph
\renewcommand{\paragraph}[1]{\oldparagraph{#1}\mbox{}}
\fi
\ifx\subparagraph\undefined\else
\let\oldsubparagraph\subparagraph
\renewcommand{\subparagraph}[1]{\oldsubparagraph{#1}\mbox{}}
\fi

%%% Use protect on footnotes to avoid problems with footnotes in titles
\let\rmarkdownfootnote\footnote%
\def\footnote{\protect\rmarkdownfootnote}

%%% Change title format to be more compact
\usepackage{titling}

% Create subtitle command for use in maketitle
\newcommand{\subtitle}[1]{
  \posttitle{
    \begin{center}\large#1\end{center}
    }
}

\setlength{\droptitle}{-2em}

  \title{Recommandations de lectures}
    \pretitle{\vspace{\droptitle}\centering\huge}
  \posttitle{\par}
    \author{David Beauchemin}
    \preauthor{\centering\large\emph}
  \postauthor{\par}
      \predate{\centering\large\emph}
  \postdate{\par}
    \date{13 septembre 2018}


\begin{document}
\maketitle

\section{Introduction}\label{introduction}

Voici une liste de recommandation de lectures pour se familiariser avec
la programmation et différents concepts clés du projet de Métrique
intergénérationnelle.

\subsection{Java}\label{java}

\begin{enumerate}
\def\labelenumi{\arabic{enumi}.}
\tightlist
\item
  \includegraphics{http://www2.ift.ulaval.ca/gaudreault/dokuwiki_a14/doku.php?id=start}
  du cours GLO-2004: Génie logiciel orienté objet

  \begin{enumerate}
  \def\labelenumii{\arabic{enumii}.}
  \tightlist
  \item
    Nom d'utilisateur: etudiants
  \item
    Mot de passe: etudiantsulaval
  \end{enumerate}
\item
  \includegraphics{http://ariane.ulaval.ca/cgi-bin/recherche.cgi?qu=a1995576}
  Chapitre: 1-8, 12-13
\end{enumerate}

\subsection{NetBeans}\label{netbeans}

\begin{enumerate}
\def\labelenumi{\arabic{enumi}.}
\tightlist
\item
  \includegraphics{https://netbeans.org/kb/trails/matisse.html} NetBeans
\item
  \includegraphics{https://codes-sources.commentcamarche.net/faq/1338-demarrage-rapide-avec-netbeans-ide}
  NetBeans
\item
  \includegraphics{https://www.youtube.com/watch?v=UTJBCMJJW_E} NetBeans
\end{enumerate}

\subsection{Compétences en
programmation}\label{competences-en-programmation}

\begin{enumerate}
\def\labelenumi{\arabic{enumi}.}
\tightlist
\item
  \emph{Clean code} - Chapitre 1 à 6
\item
  \includegraphics{https://martinfowler.com/bliki/TechnicalDebt.html} et
  son \includegraphics{https://daedtech.com/human-cost-tech-debt/} dans
  le projet
\end{enumerate}

\subsection{Autres outils}\label{autres-outils}

\begin{enumerate}
\def\labelenumi{\arabic{enumi}.}
\tightlist
\item
  \includegraphics{https://trello.com/}
\item
  \includegraphics{https://www.getharvest.com}
\end{enumerate}


\end{document}
